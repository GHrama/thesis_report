\begin{abstract}
  Data from citizens needs to be collected and analyzed to create or improve current services in society. Data collected from them, in general, reveals information about their behavior and choices. In addition, it can also reveal sensitive information, that they might not be comfortable with.
To preserve the privacy of citizens is where data privacy comes into play. There are various methods to maintain data privacy and different levels of privacy to maintain. The higher the privacy level, the more concealed the data is. Given the choice, citizens would generally choose the highest privacy level. At times, less concealed data is needed while solving problems that need data with less errors.
To help citizens reduce the level of privacy of the data when needed, different kinds of incentives can be used, such as monetary incentives. From a fixed budget on the demand side, rewards(incentives) are handed out to citizens to incite them to give less privatized data, yet maintaining a minimum level of privacy.
The goal of the Thesis is to understand the social dynamics of privacy and information sharing. Existing data can be used or data can be collected for the purpose of the analysis.

\end{abstract}
