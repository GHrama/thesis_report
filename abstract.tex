\begin{abstract}
Data from citizens needs to be collected and analysed to create or improve current services in society. Data collected from them, in general, reveals information about their behaviour and choices. In addition, it can also reveal sensitive information, that they might not be comfortable with.
To preserve the privacy of citizens is where data privacy comes into play. One of the methods to maintain data privacy is by letting citizens choose how much data to share. The lesser data that is shared or higher the privacy that is chosen, the more concealed the data is. Given the choice, citizens would generally choose the highest privacy level. At times, less concealed data is needed while solving problems that need data with less errors.
To help citizens reduce the level of privacy of the data when needed, different kinds of incentives can be used, such as monetary incentives. From a fixed budget on the demand side, rewards(incentives) are handed out to citizens to incite them to give less privatized data, yet maintaining a minimum level of privacy.
The goal of the Thesis is to understand the social dynamics of privacy and information sharing. Existing data can be used or data can be collected for the purpose of the analysis.


Today, smartphones that are being carried by billions of people have inbuilt sensors like the location, accelerometer and gyroscope. The interconnected network of smartphones where users can share large volumes of data about their behavioural and social aspects is called participatory social sensing. 


\end{abstract}
