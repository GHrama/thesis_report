\begin{abstract}
Today, there are a unrestricted opportunities for people to generate and share their data in real time with the advent of the Internet of Things, ubiquitous and pervasive computing. Due to the unstructured nature, variability and rate of growth of this data when it is recorded, analysed and processed efficiently it can lead to the understanding of one's business or competitors that leads to better products and services. Jinyan Zang et al \cite{zang2015knows} and Ashwini Rao et al \cite{rao2015they} talk about the non transparent ways of today's data collection systems where people are uninformed about the data collection process.

This Thesis builds upon the paper by Pournaras et al \cite{pournaras2016self} where data sharing is modelled as a supply-demand system. The concept of a computational market is introduced where data aggregators who analyse data can incentivize people to share their data with a certain accuracy and this is performed using summarization functions that regulate the amount of information in the data shared. 

There is a lack of information on how people make decisions in the sharing of mobile sensor data. This information is needed in order to design a platform for a computational market that rewards users fairly and increases the awareness of people about their privacy.

In this Thesis, a social experiment and an Android application are designed to examine the relationship between mobile sensor data sharing and how incentives affect this. Users are requested their data using data requests that are governed by three aspects : (a) sensor type, (b) the stakeholder to whom sensor data is shared, (c) the context or the application type for which data is shared. To do this, a computational model is introduced that assigns rewards to the possible data requests based on the user profiles formed by questioning users on the three aspects of data sharing. Data sharing decisions during the experiment are then tracked with and without incentives and the user data is recorded anonymously. The anonymous data recorded is then analysed and the findings can help in the making of a platform with fairer and more transparent data collection systems that make people more privacy aware.


\end{abstract}
