\chapter{Introduction}

In today's world, almost everybody owns a smartphone. Information collected from a large number of interconnected smartphones equipped with
multiple number of sensors each can aid to accumulate large volumes of data which is heterogeneous and autonomous in nature. This large amounts of data in any form is called Big Data\footnote{Date:23-08-2016 \url{https://en.wikipedia.org/wiki/Big\_data}} and it
is the stepping stone to large scale data analytics and Deep learning to understand the complexities in our society. 

Currently, a lot of the data collected in mobile applications\footnote{Date:23-08-2016 \url{http://www.theregister.co.uk/2014/02/21/appthority\_app_privacy\_study}} and on the web\footnote{Date:23-08-2016 \url{http://www.techlicious.com/blog/whos-gathering-your-personal-information}}
is done in a manner where users are unaware of the collection of their data. This process not only lacks transparency but also does not give users
control over their data. Users should be able to make informed decisions about what data to share or not. Additionally, since no privacy algorithms 
are implemented on the data collected, users privacies are at risk. Paul Ohm in his paper \cite{ohm2010broken} explains that it can be shown that anonymized data can be de-anonymized surprisingly easily. In this dissertation, focus is not given on the attacks such as hacking and any other methods but concentrated on the threats to the information in the data collected.

The aim is to setup a fair way to collect data by setting up a platform over a participatory sensing network where users can trade their data for some incentives to stakeholders who approach them.
Incentives can be money, vouchers or anything else. Stakeholders inform users about the date, duration for which data is collected, who will use the data and what will it be used for. Additionally, users have the possibility to share data with some added levels of privacy. 
To achieve this goal, it is first essential to understand the relationship between incentives and mobile sensor data sharing. In this dissertation, a survey and social experiment is designed where a platform is created and users can trade and view their data in a transparent manner. Data from user inputs and decisions is collected and later analysed.






