\chapter{Introduction}

%In today's world, almost everybody owns a smartphone. Information collected from a large number of interconnected smartphones equipped with
%multiple number of sensors each can aid to accumulate large volumes of data which is heterogeneous and autonomous in nature. This large amounts of data in any form is called Big Data\footnote{Date:23-08-2016 \url{https://en.wikipedia.org/wiki/Big\_data}} and it
%is the stepping stone to large scale data analytics and Deep learning to understand the complexities in our society. 
%
%Currently, a lot of the data collected in mobile applications\footnote{Date:23-08-2016 \url{http://www.theregister.co.uk/2014/02/21/appthority\_app_privacy\_study}} and on the web\footnote{Date:23-08-2016 \url{http://www.techlicious.com/blog/whos-gathering-your-personal-information}}
%is done in a manner where users are unaware of the collection of their data. This process not only lacks transparency but also does not give users
%control over their data. Users should be able to make informed decisions about what data to share or not. Additionally, since no privacy algorithms 
%are implemented on the data collected, user privacies are at risk. Paul Ohm in his paper \cite{ohm2010broken} explains that it can be shown that anonymized data can be de-anonymized surprisingly easily. In this dissertation, focus is not given on the attacks such as hacking and any other methods but concentrated on the threats to the information in the data collected.
%
%The aim is to setup a fair way to collect data by setting up a platform over a participatory sensing network where users can trade their data for some incentives to stakeholders who approach them.
%Incentives can be money, vouchers or anything else. Stakeholders inform users about the date, duration for which data is collected, who will use the data and what will it be used for. Additionally, users have the possibility to share data with some added levels of privacy. 
%To achieve this goal, it is first essential to understand the relationship between incentives and mobile sensor data sharing. In this dissertation, a survey and social experiment is designed where a platform is created and users can trade and view their data in a transparent manner. Data from user inputs and decisions is collected and later analysed to throw light on user decision of their sensor data.


Big Data systems today often collect data in ways that are unfair, non transparent and privacy intrusive to people. Most of the times, people are unaware that data collection is taking place. For this reason, new ways of acquiring data need to be designed. People should have control over their data and be given all the necessary information before data collection such as: (i) information about the data sharing process, (ii) who will access the data, (iii) what data is collected, (iv) time and duration of data collection, (v) purpose of data collection. Additionally to the above mentioned points, users should be given the choice to control the privacy or accuracy of the data they share. In addition to security threats to data, there are also threats to people's privacies due to the information content in the data shared. 

This thesis builds upon the earlier work on self-regulatory information sharing in participatory social sensing by Pournaras et al \cite{pournaras2016self}. In this concept introduced, citizens who are suppliers are the ones sharing their data and they have the choice to choose how much of their data to share. On the other side, the data aggregators are the consumers who use the sensor data shared by citizens for some data analytics task. Data aggregators require data to be of some accuracy for their task and incentivize users accordingly  to share data with a certain level of accuracy. For example, if data aggregators would like data with a higher accuracy, they can incentivize citizens with higher incentives for sharing their data. Similarly, if the analysis task does not require data with high accuracy, lower incentives can be awarded to citizens for sharing data. Information content or accuracy of data can be regulated using summarization functions concerning algorithms from simple arithmetic functions to clustering algorithms. A high summarization level filters out most of the information in the data, hence preserving the privacy of the citizen. This can cause higher errors for the data aggregator. Similarly, a lower summarization level preserves a higher information content and causes lower errors for the data aggregator.

To design a computational market platform where users can share data in a more transparent and fair manner, there is a lack of information on the choices users make in sharing their data. Additionally, there is a need to know the perception of citizens on the privacy of their mobile sensor data and this data helps understand the supply-demand system for data sharing. The findings from the above can help to create a fairer system for data collection. 

The contribution of this thesis is to bridge this gap. A social experiment is designed where citizens are approached with data requests governed by the three following aspects:

\begin{itemize}
\item The sensor type
\item The stakeholder or entity requesting for citizen's sensor data
\item The context or purpose of sensor data collection
\end{itemize} 

Each data request is assigned rewards individually for every citizen using a computational model. This model first frames a citizen profile by asking citizens questions about the three aspects of a data request mentioned above. The citizen profile is then used to assign rewards to data requests.

A social experiment is then carried out with an Android application that has an inbuilt computational model. Citizen decisions are recorded anonymously for later analysis. A pre survey is launched before the social experiment is carried out to understand the perception of users on the three aspects studied. An exit survey is also launched after the completion of the social experiment to obtain feedback from participating citizens. 

From the above, it is seen that 77.5\% of citizens are moderately or more concerned about their mobile sensor data. Also, it is found that citizens are not motivated to share their data for no incentives or because their friends do so but are open to accept incentives other than money for data sharing. Findings from the social experiment indicate that even though citizens are more interested in obtaining more rewards, ultimately the decision lies on the data request being asked. There is also an increased amount of data sharing and reduction in privacy when rewards are awarded to citizens. It is also found that citizens understood the value and privacy of their data through the rewards awarded to data requests. Additionally, it is found that 75\% of citizens visited the FairDataShare portal to view the data collected from them.  

Below is an outline of the thesis :
\begin{itemize}
\item \textbf{Chapter 2} - This chapter gives an overview of the previous work and elaborates on the drawbacks addressed in the thesis
\item \textbf{Chapter 3} - This chapter introduces the working and mathematics behind the computational model
\item \textbf{Chapter 4} - This chapter explains in detail the procedure and science behind the social experiment designed including the preliminary work
\item \textbf{Chapter 5} - This chapter explains the implementation and algorithms used in the implementation of the Android application for the social experiment designed
\item \textbf{Chapter 6} - This chapter presents the findings obtained from the surveys and the social experiment deployed
\item \textbf{Chapter 7} - This chapter concludes the thesis presenting an overview of the work done, some important findings and possible future work
\end{itemize}






