\chapter{The Fair Data Share Mobile Application}
This chapter illustrates the mobile application environment. First an overview is given, followed by a detailed explanation of the main components of the Android mobile application. This includes the architecture, database schemas and algorithms. Next, the server business logic and storage of the application is presented.

\section{The Building Blocks}

The following sections explain the integral parts of the server and client of the mobile application. A summary of the architecture is shown
in Figure \ref{fig:bb}.

\begin{figure}[ht!]
\centering
\includegraphics[width=\textwidth,keepaspectratio]{./images/blocks_app2}
\caption{Conceptual diagram of mobile application architecture}
\label{fig:bb}
\end{figure}

Users have the possibility to login onto the FairDataShare Portal from their computer or the mobile application. Once the user is authenticated, the user requests
are sent from the FairDataShare server to the Kinvey Data Store\footnote{\url{https://kinvey.com}}. Kinvey in turn fetches the appropriate data and gives it back to the FairDataShare Server. This structures the data so it is readable, and sends the data to the user to see on the portal.

\section{The Mobile Application}

The mobile application is developed for the Android platform with phones having API above level 17\footnote{\url{https://developer.android.com/guide/topics/manifest/uses-sdk-element.html}}. Phones are assumed to have internet connectivity and sufficient storage space of at least 100 Mb.
Below is an explanation of some of the tasks that take place in the application. A block diagram of the interaction of each of the components in the application is depicted in Figure \ref{fig:chap5_app}.

\subsection{Local Storage} \label{loc}
The local storage is an integral part of the application. The database used is SQLite\footnote{\url{https://developer.android.com/reference/android/database/sqlite/package-summary.html}} and is the default database
for the Android environment. Small sized unrelated data pieces are stored in preference files (as key value pairs), and larger related
data are stored in the database. The following paragraphs explain each table present in this application followed by
their function and schema. All tables explained here are pertaining to the user using the mobile application and not the server.

\begin{figure}[ht!]
\centering
\includegraphics[width=\textwidth,keepaspectratio]{./images/chap1}
\caption{Interaction of components of mobile application}
\label{fig:chap5_app}
\end{figure}

\begin{figure}[htp]
\subtop[Table schema of \texttt{QUESTION\_STORE}\label{fig:db_quest}]{\includegraphics[width=0.4\linewidth]{./images/db_quest}}\hspace{1em}
\subtop[Table schema of \texttt{WHICH\_ANSWERS}\label{fig:db_which}]{\includegraphics[width=0.4\linewidth]{./images/db_which_1}}
\caption{Table schemas}
\label{fig:ts1}
\end{figure}

Figure \ref{fig:db_quest} shows the \texttt{QUESTION\_STORE} table schema. This table stores each possible data request with its features such as with the sensor \textit{SENSOR}, stakeholder \textit{STAKEHOLDER} and context \textit{CONTEXT}. Each of these are represented by an integer, for example sensor 0 stands for accelerometer sensor. Each data request is accompanied by
an unique question identifier \textit{QID}, weight assigned \textit{WEIGHT} and the cost assigned \textit{COST}. This data is not sent to 
the server.

Figure \ref{fig:db_which} depicts the table \texttt{WHICH\_ANSWERS}'s table schema. This stores the questions identifier \textit{QID} of each data request that is answered by the user for each round. This is helpful while fetching data requests, so as not to fetch the request twice in the same round. This ensures that all questions are answered before answering them for a second time. This data is not sent to the server.

\begin{figure}[htp]
\subtop[Table schema of \texttt{STORE\_ANSWERS}\label{fig:db_ans}]{\includegraphics[width=0.4\linewidth]{./images/db_ans}} \hspace{1em}
\subtop[Table schema of \texttt{STORE\_POINTS}\label{fig:db_points}]{\includegraphics[width=0.4\linewidth]{./images/db_points}}
\caption{Table schemas}
\label{fig:ts11}
\end{figure}

Figure \ref{fig:db_ans} explains the schema of \texttt{STORE\_ANSWERS} table. This table is used to store the data request identifier \textit{QID} with the corresponding
user responses \textit{LEVEL}, along with the increase or decrease in credit obtained \textit{COST\_OBT}. The total cost is calculated by adding all the costs in this table. Similarly, the total privacy is calculated by averaging all the user responses stored in \texttt{STORE\_ANSWERS} table. Only the most recent responses to a data request are stored in this table, since data requests can be answered more than once. The content of the table is not sent to the server.

Figure \ref{fig:db_points} denotes the schema of \texttt{STORE\_POINTS} table. This table is used to store the credit and privacy obtained for each bidding day.
This information is sent to the server as soon one bidding day is over.

\begin{figure}[ht!]
\centering
\includegraphics[width=0.4\linewidth]{./images/db_ur1}
\caption{Table \texttt{USERRESPONSE\_CACHE} schema}
\label{fig:db_ur}
\end{figure}

Figure \ref{fig:db_ur} depicts the \texttt{USERRESPONSE\_CACHE} table schema. This table stores a unique key \textit{KEY} for each user response, followed by a flag \textit{ISSENT}, which is 1 if the response is not sent to the server, and 0 if it is sent. The user response saved consists of the following entries :

\begin{enumerate}
	\item User identifier
	\item Timestamp of the response
    \item Sensor identifier
    \item Stakeholder identifier
    \item Context identifier
    \item Privacy level response for this data request
    \item Cost obtained for this data request
    \item Current total privacy of the user
    \item Current total credit of the user
    \item Maximum obtainable credit for this data request in this round
    \item Metric chosen to improve  (Improve privacy or improve credit)
\end{enumerate}

All of the above fields are packed into the field \textit{ur} shown in Figure \ref{fig:db_ur}. The data in this table is sent to the server. Once the entry is sent to the server, the \textit{ISSENT} field is changed to 0 and deleted locally. The unique keys \textit{KEY} are useful for deleting sent entries. Figures \ref{fig:ts2} and \ref{fig:ts22} show the table schemas for data storage of the following sensors:

\begin{enumerate}
	\item Accelerometer in the \texttt{STORE\_ACCELEROMETER}
	\item Noise in the \texttt{STORE\_NOISE}
    \item Location in the  \texttt{STORE\_LOCATION}
    \item Light in the  \texttt{STORE\_LIGHT}
\end{enumerate}

The general schema for all the sensor tables is the following :

\begin{enumerate}
	\item \textit{KEY} - Uniquely identifies each sensor entry
	\item \textit{TIMESTAMP} - The time when the sensor value is collected
    \item \textit{ISSENT} - Denotes whether the sensor entry is already sent to the server or not
    \item The other columns are specific to each sensor and represent the actual sensor values collected 
\end{enumerate}

\begin{figure}[htp]
\subtop[Table schema of \texttt{STORE\_ACCELEROMETER}\label{fig:db_acc}]{\includegraphics[width=0.4\linewidth]{./images/db_acc}}\hspace{1em}
\subtop[Table schema of \texttt{STORE\_NOISE} \label{fig:db_noise}]{\includegraphics[width=0.4\linewidth]{./images/db_noise}}%
\caption{Table schemas for sensor data}
\label{fig:ts2}
\end{figure}



\begin{figure}[htp]
\subtop[Table schema of \texttt{STORE\_LOCATION}\label{fig:db_loc}]{\includegraphics[width=0.4\linewidth]{./images/db_loc}}\hspace{1em}
\subtop[Table schema of \texttt{STORE\_LIGHT} \label{fig:db_light}]{\includegraphics[width=0.4\linewidth]{./images/db_light}}%
\caption{Table schemas for sensor data}
\label{fig:ts22}
\end{figure}

\subsection{Alarms and Notifications} \label{alarm}

Every bidding day lasts 24 hours, and in this period the user answers data requests. After the completion of one bidding day, the system needs to be informed in a timely manner to perform some application critical functions such as going to the next bidding day. To inform the system of such an event, Android provides this functionality in the form of alarms. 

Alarms can be set to go off just once or repeatedly to trigger tasks.
The built-in repetitive alarms\footnote{\url{https://developer.android.com/training/scheduling/alarms.html}} provided are not uniform across different Android versions because they are optimized to save battery. This can cause a delay of upto 24 hours in triggering the alarm. Hence, the application is programmed to set repeating alarms manually. 

The first time the application opens, the alarm is set to ring in exactly 24 hours, but the timing changes when the phone is switched off.
One of the conditions of the experiment is not to have the phone switched off at any time. Nevertheless, it is taken into account the scenario where
the phone is kept switched off for a period of time. There are various things that can happen:

\begin{enumerate}
	\item The phone is rebooted.
	\item The phone is switched off, during this time an alarm is missed.
    \item The phone is switched off for a period greater than 24 hours. One or more alarms can be missed.
\end{enumerate}

Once the phone is switched off, all alarms are erased from memory\footnote{\url{https://developer.android.com/reference/android/app/AlarmManager.html}} and alarms do not execute when the phone is switched off. Hence, when the phone switches on the
BootReceiver service of the application is triggered with pseudocode shown in Algorithm \ref{boot}. This checks whether an alarm has been missed, 200 seconds are given for the phone to stabilize after booting before triggering tasks. Otherwise, a new alarm is set using the pseudocode shown in Algorithm \ref{setalarm}. To set an alarm the time needed is the difference between the current time and when the alarm should be triggered. After that is calculated, the alarm is set. 86400 in the pseudocode indicates the number of seconds in a 24 hour period and is used to check if it has been more than 24 hours since an alarm has been triggered.

\begin{algorithm}
\caption{BootService Algorithm}\label{boot}
\begin{algorithmic}[1]
\Procedure{BootService}{}
\State $\textit{now} \gets \text{current timestamp}$
\State $i \gets \text{timestamp of last triggered alarm}$
\If {$\textit{now}-i < 86400$}
  \State $\text{Call }\textit{SetAlarmLater()}$
\Else
  \State $\text{Set alarm in 200 seconds}$
\EndIf
\EndProcedure
\end{algorithmic}
\end{algorithm}

The alarm Algorithm \ref{setalarm} is used to set the next alarm after the phone boots or when an alarm has just rung. It sets the alarm exactly 24 hours after the last alarm rang.

\begin{algorithm}
\caption{Alarm Algorithm}\label{setalarm}
\begin{algorithmic}[1]
\Procedure{SetAlarmLater}{}
\State $\textit{now} \gets \text{current timestamp}$
\State $i \gets \text{timestamp of last triggered alarm}$
\State $\textit{latertime} \gets \textit{i}+\text{86400}$
\State $\textit{latergap} \gets \textit{latertime}-\textit{now}$
\State $\text{Set Alarm in latergap seconds}$
\EndProcedure
\end{algorithmic}
\end{algorithm}

\subsubsection{Going to the Next Data Sharing Day} \label{next}
Once the alarm rings, it marks the end of a bidding day. Once a bidding day ends, a number of tasks need to be executed
and for this the NextDayService is triggered, which is described in pseudocode shown in Algorithm \ref{nextday}. Firstly, the privacy and credit is sent to the server and stored locally in the \texttt{STORE\_POINTS} table. \textit{Privacy} which is the total privacy obtained, \textit{Credit} is the total credit obtained, \textit{Round} which is the number of times the user answers all the questions and \textit{CurrentQuestion} which is the current question the user is answering is all reset to zero. The \textit{Day} corresponds to the current bidding day which increments by one to denote the next bidding day.

\begin{algorithm}
\caption{NextDayService Algorithm}\label{nextday}
\begin{algorithmic}[1]
\Procedure{NextDayService}{}
\State $\text{Store }\textit{Privacy, Credit, Day } \text{in } \textit{STOREPOINTS}$
\State $\text{Send }\textit{Privacy, Credit, Day } \text{to Server}$
\State $\textit{Privacy, Credit, Round, CurrentQuestion} \gets \text{0}$
\State $\textit{Day} \gets \textit{Day}+1$
\State $\text{Store current time}$
\State $\text{Call }\textit{Summarization()}$
\If {$\textit{Day} > \textit{End}$}
  \State $\text{End experiment}$
\Else
  \State $\text{Update user interface elements}$ 
\EndIf
\EndProcedure
\end{algorithmic}
\end{algorithm}

The current time of executing the alarm is saved in case the phone is rebooted or switched off as mentioned in Section \ref{alarm}. After that, the sensor data which is saved locally
needs to be summarized, the corresponding method is called and is explained in pseudocode shown in \ref{sum}. A final check is done to see if the experiment is done and the user interface is updated accordingly. This means either the various metrics on the improvement and bidding screens (which ever is currently active) are updated, or
the end of experiment screen is shown.

\subsection{Fetching Data Requests} \label{data_req}
A data request needs to be fetched from the mobile phone database in two scenarios :

\begin{enumerate}
	\item After a question is answered in the first bidding day (entry phase)
	\item After the privacy or credit improvement button is clicked (core phase)
\end{enumerate}

In the first bidding day, once a data request is answered the next is fetched sequentially from the mobile phone database. This requires knowing the current data request number and fetching the next data request from table \texttt{QUESTION\_STORE}. For the other bidding days, fetching of the data requests depends on the improvement button chosen. According to the choice, the following is done:

\begin{enumerate}
	\item \textbf{Improve Privacy} - Obtain the data request from table \texttt{STORE\_ANSWERS} where the user can maximize the privacy metric
	\item \textbf{Improve Credit} - Obtain the data request from table \texttt{STORE\_ANSWERS} where the user can maximize the cost metric
\end{enumerate}

In addition to sending the data request to the user interface, it is needed to show how choosing each option of the data request will affect the total privacy and total credit metrics. To do this for the total cost, the computation $last-possible$ is output, where $last$ stands for the credit obtained the last time the data request is answered. $possible$ stands for the maximum amount of credit that can be obtained for this option (each data request has five privacy options \ref{o}). The possible total cost changes are shown under the options. For more detail on how credits are split among options in a data request refer to Section \ref{options}.

Every option of a data request has an associated percentage of data that is shared as described in Section \ref{o}. According to the percentage of data shared, the total privacy is calculated for each possible option. The difference between the current privacy and each potential privacy (the privacy if the user chooses to click a particular option) is calculated and indicated under each option. This gives an indication to the user as to how each option will affect the privacy and cost metrics.

\subsection{Recording User Choices}

Figure \ref{fig:db_ur} describes the table \texttt{USERRESPONSE\_CACHE}. Each time a user enters a response to a data request, all the fields mentioned in Section
\ref{loc} are recorded and stored in a class object. This object is transformed into a byte array.
When the JobNetworkService described in Section \ref{job} is called, the byte array is converted back into a class object and sent to the Kinvey Data Store.

\subsection{Sensor Data Collection and Summarization} 

Sensor data is collected from the following sensors :

\begin{enumerate}
	\item Accelerometer sensor
	\item Noise sensor
    \item Location sensor (GPS)
    \item Light sensor
\end{enumerate}

A sensor service is triggered when the application is installed and is stopped when the experiment is over. The sensor service collects data from all sensors
every 30 seconds and stores it in the appropriate tables mentioned in Section \ref{loc}.
At the end of a bidding day, sensor data needs to be summarized according to the privacy level chosen by the user. This starts by first finding out the lowest privacy level for each sensor. Privacy levels range from one to five, that is from the lowest to highest privacy levels. Using this level,
summarization is done as shown in pseudocode \ref{sum}. Every privacy level corresponds to an action:

\begin{enumerate}
	\item 1- Send all data to the server
	\item 2- Send 75\% of the data
    \item 3- Send 50\% of the data
    \item 4- Send 25\% of the data
    \item 5- Do not send any data
\end{enumerate}

Initially all the sensor data has a field \textit{ISSENT} with value of zero. Data to be sent to the server is set with $\textit{ISSENT}=1$, and all others that have value $\textit{ISSENT}=0$ are ignored. In the pseudocode \ref{sum}, first the privacy level to which summarization should be performed is obtained. If the privacy level is 5, then no records are sent to the server and hence all data remains marked with $\textit{ISSENT}=0$. If the privacy level is 1, all records should be sent to the server and they are all hence marked with $\textit{ISSENT}=1$ without any summarization performed. Similarly, if the privacy level is 2, 3 or 4 the amount of records to be sent to the server are 75\%, 50\% and 25\% respectively. To do this the, amount of records(mobile sensor data) sent to the server is 3 records for every 4 records, 1 record for every 2 records and 1 records for every 4 records respectively and they are marked with $\textit{ISSENT}=1$.

\begin{algorithm}
\caption{Summarization Algorithm}\label{sum}
\begin{algorithmic}[1]
\Procedure{Summarization}{}
\For{$\text{each sensor}$}
\State $\text{Fetch sensor data from } \textit{sensor table}$
\State $\textit{level} \gets \text{Fetch user privacy level}$
\If {$\textit{level} \gets 1$}
  \State $\text{Set all } \textit{ISSENT} \gets \text{1}$
\ElsIf {$\textit{level} \gets 2$}
	\For{$\text{3 out of every 4 records}$}
 	 \State $\textit{ISSENT} \gets \text{1}$
 	\EndFor 
\ElsIf {$\textit{level} \gets 3$}
  \For{$\text{1 out of every 2 records}$}
 	 \State $\textit{ISSENT} \gets \text{1}$
 	\EndFor
\ElsIf {$\textit{level} \gets 4$}
  \For{$\text{1 out of every 4 records}$}
 	 \State $\textit{ISSENT} \gets \text{1}$
 	\EndFor
\EndIf
\State $\text{Delete all entries with } \textit{ISSENT} \gets 0$
\State $\text{Update Database}$
\EndFor
\EndProcedure
\end{algorithmic}
\end{algorithm}

\begin{algorithm}
\caption{JobNetworkService Algorithm}\label{nextday}
\begin{algorithmic}[1]
\Procedure{NetworkService}{}
\State $ \text{Fecth data from } \textit{USERRESPONSECACHE}$
 \For{$\text{each record}$}
 	 \If {$\textit{ISSENT} == \textit{1}$}
  \State $\text{Send record to Server}$
  \If {$\text{SUCCESS}$}
  \State $\text{Delete record}$  
  \EndIf
  \EndIf
 	\EndFor

\For{$\text{each sensor}$}
 	 \State $ \text{Fecth data from } \textit{sensor table}$
 	  \For{$\text{each record}$}
 	 \If {$\textit{ISSENT} == \textit{1}$}
  \State $\text{Send record to Server}$
  \If {$\text{SUCCESS}$}
  \State $\text{Delete record}$  
  \EndIf
  \EndIf
 	\EndFor
	
 	\EndFor
\EndProcedure
\end{algorithmic}
\end{algorithm}


\subsection{Server Synchronization} \label{job}

User responses and sensor data are sent to the server. This is done periodically every 5000 seconds in order to empty the space on the phone whenever internet is available. It is triggered first when the application is started for the first time. Data is fetched from the tables in the database. Data with fields marked as $\textit{ISSENT}=1$ is data that is ready and that has not been sent to the server. Such data is sent, and when an acknowledgement is received from the server, this data is deleted from the table.

\section{The Server}

\subsection{Kinvey Data Storage}

Kinvey\footnote{\url{http://kinvey.com/}} is a mobile backend as a service which provides a platform for mobile phones to link applications to a backend cloud storage. For the purpose of this application, the backend has is used to store data and for some business logic implementations in javascript.

\subsubsection{Security}

All communications from the application to the server are encrypted using TLS/SSL encryption\footnote{Kinvey white paper : KINVEY CLOUD
SERVICE: SECURITY
OVERVIEW 2014}, to communicate with the backend service. This is automatically provided and done by the Kinvey SDK.

\subsubsection{Collection Store}

Locally, all information is stored in SQLite which is a relational database. The database used by the Kinvey Data Store is MongoDB. The schema of the collections in Kinvey are the same as described in Section \ref{loc} for the mobile phone databases.
When the user starts the application, general personal information is entered as explained in Section \ref{loc}. This data is stored in the
collection USERINFORMATION with the screenshots shown in the Figures \ref{fig:col_ui_1} and \ref{fig:col_ui_2} and schema in Figure \ref{fig:cat_cat_cat}.

\begin{figure}[ht!]
\centering
\includegraphics[width=0.4\linewidth]{./images/schema_ui}
\caption{Table \texttt{USERINFORMATION} schema}
\label{fig:cat_cat_cat}
\end{figure}

Once this is done, users have to categorize the various features, sensors, stakeholders and then the various contexts. This information is sent to the server in collections named FEATURES, SENSORS, STAKEHOLDERS and CONTEXTS. Screenshots are shown in Figures \ref{fig:col_f}, \ref{fig:col_s}, \ref{fig:col_ss} and \ref{fig:col_c} respectively and the schemas in Figure \ref{fig:cat_cat}. User responses are stored in the collection UserResponse shown in Figures \ref{fig:col_ur_1} and \ref{fig:col_ur_2} and their schema is shown in Section \ref{loc}.

\begin{figure}[htp]
\subtop[Table schema of \texttt{FEATURES}\label{fig:db_f}]{\includegraphics[width=0.4\linewidth,height=0.4\linewidth]{./images/schema_features}}\hspace{1em}
\subtop[Table schema of \texttt{SENSORS} \label{fig:db_s}]{\includegraphics[width=0.4\linewidth]{./images/schema_sensors}}\newline
\subtop[Table schema of \texttt{STAKEHOLDERS}\label{fig:db_ss}]{\includegraphics[width=0.4\linewidth]{./images/schema_stakeholders}}\hspace{1em}
\subtop[Table schema of \texttt{CONTEXTS} \label{fig:db_c}]{\includegraphics[width=0.4\linewidth]{./images/schema_contexts}}%
\caption{Table schemas for categorizations}
\label{fig:cat_cat}
\end{figure}

The sensor data sent by the JobNetworkService is stored in collections named after the sensors themselves. The screenshots of the tables
are shown in Figures \ref{fig:col_loc}, \ref{fig:col_acc}, \ref{fig:col_light} and \ref{fig:col_noise} and their schema is shown in Section \ref{loc}.

To keep track of all the existing users in the experiment, the collection USERS stores all unique user identification strings of participants.
The screenshot is shown in Figure \ref{fig:col_users}.

Finally, the collection SCORE shown in Figure \ref{fig:col_score} and the schema is shown in Figure \ref{fig:db_points} stores the total privacy and total credit obtained by the user for each bidding day.

\subsubsection{Bussiness Logic} \label{bl}
Most of the bussiness logic used for the FairDataShare portal is present in Kinvey. There are two main scripts stored in Kinvey:

\begin{enumerate}
    \item Script to find the privacy preferences of users
    \item Script to perform data summarization
\end{enumerate}

The stakeholders make a request for data on the FairDataShare portal giving the following details:

\begin{enumerate}
    \item Bidding day number
    \item Anonymous user
    \item Sensor
    \item Context
\end{enumerate}

Given this input and the category of the stakeholder (which is known from their registration), we look into the UserResponse Collection trying to find the most recent record that
fits these criteria and extract the privacy level (this is the level to which sensor data should be summarized for this request).

Once the privacy level is known, summarization on user data is performed. Data is sent from the user phones with a certain summarization level from each sensor in order to avoid sending sensor data several times to the server. 

If the summarization level with which data is collected from the user is lower than the privacy level needed, further summarization needs to be performed. When further summarization needs to be done, the difference between the privacy level and summarization level times hundred is the percentage of data that needs to be removed for this request. The pseudocode is shown in Algorithm \ref{sum1}.

\begin{algorithm}
\caption{Server Summarization Algorithm}\label{sum1}
\begin{algorithmic}[1]
\Procedure{Summarization}{}
\State $\textit{data} \gets \text{sensor data from collection}$
\State $\textit{sl} \gets \text{level with which data was summarized on the phone}$
\State $\textit{pl} \gets \text{level to which data needs to be summarized}$
\If {$\textit{sl} == \textit{pl}$}
	\State $\textbf{Return }\textit{data}$
\Else
	\State $\textit{skip} \gets \textit{sl}-\textit{pl}+1$
	 \For{$\text{every }\textit{skip } \text{number records out of 4}$}
 	 \State $\text{Delete record from}\textit{ data}$
 	 \EndFor
\EndIf
\State $\text{Return data to portal}$
\EndProcedure
\end{algorithmic}
\end{algorithm}

\subsection{FairDataShare Web Portal}
The FairDataShare portal makes use of a server at ETH Zurich other than the Kinvey Data Store to store the usernames, passwords of the users and the stakeholders in a collection. The database technology used is MongoDB. The language used to interact with Kinvey is Express.js, which is based on Node.js. Most of the data portal business logic is on Kinvey as described in section \ref{bl}. The webpage is constructed using simple Html and css. All screenshots of the portal including detailed information are provided in Chapter \ref{exp}.









