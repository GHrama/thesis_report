\chapter{Conclusion and Future Work}

A computational model is introduced to be able to assign incentives to mobile sensor data requests. A user profile is formed and using this profile incentives are assigned in a way that can increase mobile sensor data sharing. This means that more incentives are assigned to data requests where users are more reluctant to share mobile sensor data. The model additionally includes the option for users to choose data requests according to the metric (privacy or cost) they wish to improve. This model is incorporated in a mobile application called FairDataShare mobile app. The application shows data requests where stakeholders can request users to share their mobile sensor data for a particular context. Users can then choose from five different privacy options ranging from giving all of their data (level 1) to giving none at all (level 5). 

All inputs to the application and sensor data shared by users are recorded and sent to the server by the mobile application background services. Additionally, a pre-survey is deployed to understand the perception of users on mobile sensors, stakeholders and contexts. This is also used to reduce the number of sensors, stakeholders and contexts in data requests and choose which ones to examine in more detail. Users can access the FairDataShare website to see the data that has been collected from them. Furthermore, stakeholders can also view the data shared by users with the appropriate privacy level chosen by the users themselves. This makes data sharing a more transparent process where users are requested for data and have the ability to choose from various privacy options rather than an all or none option. Furthermore, users can view the data that is collected from them on the FairDataShare portal which makes the whole data sharing process more transparent and gives users more control of their data.

From the pre survey, it is seen that 77.5\% of users are at least moderately and more concerned about their mobile sensor data. It is also observed that users are not motivated to share their data for no incentives at all, which means that incentives play an important role in the data sharing process. An emulation of the social experiment with 9 participants where no incentives were awarded, was held and it is seen that the mobile application and the FairDataShare web portal are fully functional. Additionally, it is observed that there is an increase in data sharing on the days where incentives were given compared to the days where no incentives were given. 

In the future, the social experiment will be held with a larger sample population who are awarded the incentives indicated in the mobile application with the help of the ETH Decision Laboratory. More work can be done to analyse the data obtained to find inter-relationships between features and relate the data to the user information provided. Deeper comparisons of the pre survey and experiment data can also be done. Furthermore, the model could incorporate machine learning algorithms to predict the sequence of user choices based on previous ones to assign appropriate incentives for each data request. 
