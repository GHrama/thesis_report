\chapter{Conclusion and Future Work}

A computational model is introduced to be able to assign incentives to mobile sensor data requests. A user profile is formed and using this profile rewards are assigned to data requests in a personalized manner. The model additionally includes the option for users to choose data requests according to the metric (privacy or cost) they wish to improve. This model is incorporated in a mobile application and launched on Google Play Store. The application shows data requests where stakeholders can request users to share their mobile sensor data for a particular context. Users can then choose from five different privacy options ranging from giving all of their data (level 1) to giving none at all (level 5). 

All inputs to the application and sensor data shared by users are recorded and sent to the server by mobile application background services. Additionally, a pre-survey is deployed to understand the perception of users on mobile sensors, stakeholders and contexts. This is also used to reduce the number of sensors, stakeholders and contexts in data requests and choose which ones to examine in more detail. Users can access the FairDataShare website to see the data that is collected from them. Furthermore, stakeholders can also view the data shared by users with the appropriate privacy level chosen by the users themselves. This makes data sharing a more transparent process where users are requested for data and have the ability to choose from various privacy options rather than an all or none option. Furthermore, users can view the data that is collected from them on the FairDataShare portal which makes the whole data sharing process more transparent and gives users more control of their data.

From the data obtained in the pre survey, it is seen that 77.5\% of users are at least moderately concerned about their mobile sensor data. It is also observed that users have lower motivation to share their data for no incentives, which means that incentives play an important role in the data sharing process. Additionally, it is revealed by users that money is not the only incentive that would be accepted. The GPS, camera, microphone and bluetooth sensors are found to be privacy intrusive. Corporation and government stakeholders are found to be privacy intrusive. Finance, health, shopping and social networking contexts are found to be most privacy intrusive.

An emulation of the social experiment with 9 participants where initially no incentives are awarded, is held and it is seen that the mobile application and the FairDataShare web portal are fully functional. Additionally, it is observed that there is an increase in data sharing on the days where incentives are given compared to the days where no incentives are given. It is also observed that for most cost and privacy metric values, users tend to click on the "improve credit" button. It is also seen that even though users click on the "improve credit" button, the ultimate decision to improve the cost or privacy metric depends on the data request itself.

From the exit survey, it is seen that 75\% of users find the quality of the application to be moderate (not good or bad) and above. 75\% of users find the application easy to use. Users rate the performance of the application to a level of 3.63 on 5, but rate the number of questions and battery life to be 2.13 and 2 respectively, which means that users find that there are too many questions in the experiment and their phone battery is affected. Users find the total cost and each option reward of a data request as the most useful and comprehensible features of the application. Additionally, users are not as satisfied with the total available budget of 30 CHF but they are more satisfied with the amount of rewards obtained out of the total which means they could serve their goal.

It is also seen that users are willing to sacrifice their privacy for rewards. Another outcome is that the experiment makes users aware about the privacy of their data. The rewards also cause privacy awareness and also cause awareness about the value of users data. During the experiment 25\% of users wanted to drop out at some point whereas 75\% of users were still interested to continue. 75\% of users visited the FairDataShare portal. 66.67\% of users also faced problems with battery drain.

In the future, the social experiment will be held with a larger sample population who are awarded the incentives indicated in the mobile application with the help of the ETH Decision Laboratory. More work can be done to analyse the data obtained to find inter-relationships between features and relate the data to the user information provided. Deeper comparisons of the pre survey and experiment data can also be done. Furthermore, the model could incorporate machine learning algorithms to predict the sequence of user choices based on previous ones to assign appropriate incentives for each data request. Additionally, problems mentioned by users during the exit survey will be addressed. Another interesting addition would be to increase the duration of the data collection and see the behavioural change in users.
