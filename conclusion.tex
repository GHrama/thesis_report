\chapter{Conclusion and Future Work}

In this dissertation, a survey was deployed to understand the perception of users on mobile sensors, stakeholders and contexts. This information was analysed and used in fine tuning the parameters of the mobile application. The mobile application developed has a computational model inbuilt that automatically assigns incentives to data requests according to the user profiles. A data request consists various features such as the sensor type to collect, stakeholder who is collecting the data and the purpose of data collection. The model assigns higher incentives to data requests which are considered intrusive and lesser incentives to data requests that are found less intrusive. The application collects all user input data plus mobile sensor data shared with the summarization chosen by the user. This data is sent to the cloud server and is accessed by both the user and the stakeholder in a transparent manner. Interrelationships within features are examined and the data obtained from the mobile application is analysed. Results to reinforce the need for the computational model are obtained. Additionally, an increased amount of data sharing is obtained after users are compensated with incentives.

In future, the social experiment can be held with a larger more representative pool of people who are actually awarded the incentives indicated in the mobile application. More work can be done to analyse the data obtained to find inter-relationships between features and relate the data to the user information provided. Deeper comparisons of the pre survey and experiment data can also be done. Furthermore, the model could incorporate machine learning algorithms to predict the sequence of user choices based on previous ones to put up appropriate incentives for each data request.
